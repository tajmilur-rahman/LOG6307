\documentclass[10pt, conference]{IEEEtran}
\usepackage[english]{babel}
\usepackage[usenames]{color}
\usepackage{colortbl}
\usepackage{comment}
\usepackage{graphicx}
\usepackage{epsfig}
\usepackage{array, colortbl}
\usepackage{listings}
\usepackage{epstopdf}
\usepackage{multirow}
\usepackage{rotating}
%\usepackage{subfigure}
\usepackage{subfig}
\usepackage{float}
\usepackage[obeyspaces,hyphens,spaces]{url}
\usepackage{balance}
\usepackage{fancybox}
\usepackage{scalefnt}
\usepackage[normalem]{ulem}
%\pagestyle{plain}
\pagenumbering{arabic}
\pagestyle{empty}
\clubpenalty = 10000
\widowpenalty = 10000
\displaywidowpenalty = 10000
\usepackage{cleveref}

\makeatletter
\renewcommand{\paragraph}[1]{\noindent\textsf{#1}.}

\title{An Empirical Study for Prioritizing Quality Assurance : A Replication Work}
\author{Md Tajmilur Rahman$^{1}$, Loui Philippe Querel$^{1}$
    \\
	\emph{$^{1}$ Dept. of Computer Science and Engineering, Concordia University, Qu\'{e}bec, Canada}
}

\begin{document}
\maketitle

\begin{abstract}
Lorem ipsum dolor sit amet, consectetur adipiscing elit. Nam nibh nisi, ultricies a placerat id, pharetra quis arcu. Donec ut rhoncus odio, in luctus turpis. Praesent in tellus in tellus volutpat sagittis non in felis. Praesent commodo, nisl ac ornare porta, quam libero consectetur mi, sed facilisis elit enim non ipsum. Ut consequat eros id ultricies iaculis. Ut pellentesque rhoncus neque. Integer vestibulum ac diam vitae faucibus. Sed sit amet viverra enim. Suspendisse eu nulla vel turpis auctor posuere sit amet non metus.
\end{abstract}

\section{Introduction}
\label{sec:introduction}

Quality assurance in software development plays the prominent role for a software to turn it into a successful business product. Doing business with a software product by providing services or by selling the software itself demands high level of quality assurance and quality can be assured in software industries by unit test, black-box, white-box, integration, functional and system level testing. Testing is not just the end of the quality assurance but the start of working for assuring quality of a software system. Testing produces list of regressions and flaws throughout the system that keeps the development team busy for fixing bugs to make the system flaw-less. Generally this fixing and testing be practised during the stabilization period once all the new feature development works are done. We can estimate development time and cost prior to a development cycle but the effort that stabilization period consumes for fixing does influence the cost of software product and reputation of the software industry.

In current trend many researchers are focusing on predicting defects in the code-base of the software system~\cite{Gyimothy2005IEEE}. In some contexts this approach can be useful but they have drawbacks. Predicting bugs does not specify the amount of work and cannot minimize the effort in a great deal but comparatively predicting the particular change that could be defect inducing would be more efficient to reduce the effort for fixing bugs around the time of release. This is really important for reducing the stabilization time and releasing the software product sooner to beat competitors in the market.

\section{Background and Related Work}
\label{sec:backgr-relat-work}

Phasellus laoreet ipsum non nunc sodales molestie. Aliquam rutrum urna ante, at dictum odio dictum in. Quisque sit amet lorem non mi adipiscing aliquam. Suspendisse potenti. Aenean congue a risus vel posuere. Vestibulum tempor commodo ipsum vitae congue. Nunc vestibulum volutpat sapien quis tincidunt. Vestibulum vitae ullamcorper eros. Integer luctus quam risus~\cite{humble10}. Suspendisse scelerisque nulla nulla, sed ullamcorper enim faucibus sed. Curabitur bibendum ipsum quis justo tincidunt, et pulvinar enim ullamcorper. Pellentesque et tempor turpis. Pellentesque vel nisi metus. Proin laoreet vehicula vestibulum. Vivamus iaculis urna velit, et pharetra risus scelerisque quis~\cite{baysal11}.

\section{Approach}
\label{sec:approach}

\section{Results}
\label{sec:results}

\section{Comparison of Results}
\label{sec:comparison-results}

\section{Conclusion}
\label{sec:conclusion}

Sed ullamcorper augue a lectus mollis gravida. Aliquam in commodo tortor, eget dignissim velit. Phasellus suscipit felis non nisl consequat, quis tempus sapien volutpat. Pellentesque a sagittis lectus. Nunc quis pulvinar velit, quis auctor nisl. In sed erat lectus. Vivamus eget justo et urna consequat consequat. Praesent id nisl odio. Vestibulum aliquam sit amet risus vel pretium. Aenean blandit diam at sem vulputate, sed lobortis magna vulputate. Curabitur nisi velit, tempor ut elit non, aliquet tristique sem. Vestibulum ante ipsum primis in faucibus orci luctus et ultrices posuere cubilia Curae; Vestibulum quis sollicitudin libero.

\section*{Acknowledgment}
TODO

\balance
\bibliographystyle{IEEEtran}
\bibliography{assignment.bib}
\end{document}
